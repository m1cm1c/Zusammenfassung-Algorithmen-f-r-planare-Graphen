\documentclass[a4paper,11pt]{report}
\usepackage[T1]{fontenc}
\usepackage[utf8]{inputenc}
\usepackage[german]{babel}
\usepackage{lmodern}
\usepackage{mathtools}
\usepackage{graphicx}
\usepackage{grffile}
\usepackage{float}
\usepackage[a4paper, total={6in, 10in}]{geometry}
\usepackage[colorlinks]{hyperref}
\usepackage{csquotes}
\usepackage[acronym,toc]{glossaries}
\usepackage{enumitem}
\usepackage{listings}
\usepackage[section]{placeins}
\usepackage{amsmath}
\usepackage{amssymb}
\usepackage{parskip}

\title{Zusammenfassung Algorithmen für planare Graphen}
\author{Christoph Michelbach}


\makeatletter
\hypersetup{
    colorlinks,
    citecolor=black,
    filecolor=black,
    linkcolor=blue,
    urlcolor=black,
    pdftitle={\@title},
    pdfauthor={\@author},
    bookmarks=true
}

\newcommand{\fig}[2]{
    \begin{figure}[h]
        \begin{center}
            \includegraphics[width=#2\textwidth]{Resources/Included Graphics/#1}
            \caption{#1}
            \label{fig:#1}
        \end{center}
    \end{figure}
}

\newcommand{\glsquotation}[1]{
    \paragraph{\gls{#1}:} \glsentrydesc{#1}.
}


\makeglossaries


\newglossaryentry{exampleEntry}
{
    name=example entry,
    plural=example entries,
    description={This is an example glossary entry},
}


\begin{document}
\maketitle
\tableofcontents

\chapter{Einführung}
$K_a$ ist ein vollstd. verbundener Graph aus $a$ Knoten.\\

$K_{a, b}$ ist ein Graph, dessen Knoten aus disjunkten Mengen $M_a$ und $M_b$ der Kardinalitäten $a$ bzw. $b$ bestehen, so dass alle Knoten aus $M_a$ mit allen Knoten aus $M_b$ verbunden sind, es aber keine keine weiteren Verbindungen gibt.\\

Einen Graphen auf Planarität zu prüfen, ist in $\mathcal{O}(n)$ möglich. Das liegt mit daran, dass die Anzahl der Kanten von planaren Graphen in $\mathcal{O}(n)$ liegt.\\

\paragraph{Satz von Koratowski} Ein Graph ist planar, gdw. er weder $K_5$ noch $K_{3, 3}$ als Teilgraphen enthält.\\

Die Einbettungen von Graphen sind verschieden, gdw. ihre Fasetten verschieden sind.
\fig{Bsp. fuer untersch. Fasetten: Die Fasette der oberen und der linken unteren Darst. sind gleich, die der rechten unteren aber nicht}{.7}

\paragraph{Jordan-Kurven} sind stetige, sich selbst nicht kreuzende Kurven.\\


\section{Dualgraph $G^*$}
Der Dualgraph zum pl. Graph $G = (V, E)$ mit pl. Einbettung $\mathcal{F}$ ist $G^* = (V^*, E^*)$, wobei für jede Fasette $f \in \mathcal{F}$ ein Knoten $v_f \in V^*$ existiert und für jede Kante $e \in E$ eine Kante $e^* \in E^*$, die die beiden Knoten in $G^*$ verbindet, die zu den beiden Fasetten in $\mathcal{F}$ gehören, an die $e$ angrenzt bzw. den Knoten mit sich selbst verbindet, der zu der Fasette gehört, an die $e$ angrenzt.\\

\begin{itemize}
    \item $G^*$ ist auch planar.
    \item $G^*(G^*) = G$ (bzgl. kanonischer pl. Einbettung von $G$ bzw. $G^*$.
    \item $G^*$ kann Multigraph sein.
\end{itemize}

Für pl. Graphen $G$ mit bel. pl. Einbettung $\mathcal{F}$ gilt: Ein {\bf Schnitt} (als Kantenmenge) in $G$ entspricht einem {\bf Kreis} in $G^*$.


\section{Planare Einbettung}
Eine planare Einbettung zerlegt $\mathbb{R}^2$ in {\bf Fasetten} (Gebiete, Flächen). Die Fasettenmenge wird mit $\mathcal{F}$ bezeichnet. Die planare Einbettung ist bereits durch ihre Fasettenmenge bzw. die Reihenfolge der Kurven um jede Fläche bzgl. Reihenfolge der Adjazenzlisten der Knoten festgelegt.


\section{Kombinatorische Einbettung}
Die kombinatorische Einbettung steht gegenüber der {\bf geometrischen Einbettung} (mit konkreten Koordinaten der Punkte).


\section{Satz von Euler (1750)}
$G = (V, E)$ ist ein pl. Graph mit bel. pl. Einbettung $\mathcal{F}$. $n \coloneqq |V|$, $m \coloneqq |E|$, $f \coloneqq |\mathcal{F}|$\\

In einem zusammenhängenden, planaren, einfachen (nicht Multigraph), nichtleeren Graphen gilt für jede planare Einbettung: $n - m + f = 2$


\chapter{Lemmas}
\begin{itemize}
    \item Ein pl. Graph mit mind. 3 Knoten hat höchstens $3n-6$ Kanten.
    \item Ein maximaler pl. Graph (ein Graph, zu dem man keine Kante hinzufügen kann, ohne Planarität zu verletzen) mit $n$ Knoten, hat die max. Anzahl an Knoten unter allen pl. Graphen mit $n$ Knoten.
    \item Sei $G$ pl. mit mind. 3 Knoten. $d_{max}(G)$ bezeichne den Maximalgrad in $G$ und $n_j$ bezeichne die Anzahl Knoten in $G$ mit Grad $i$. Dann gilt: \[6 n_0 + 5 n_1 + 4 n_2 + 3 n_3 + 2 n_4 + n_5 \geq n_7 + 2 n_8 + 3 n_9 + ... + (d_{max}(G) - 6) \cdot n_{d_{max}(G)}\]
\end{itemize}




\glsaddallunused
\printglossaries

\end{document}


%%% Local Variables:
%%% mode: latex
%%% TeX-master: t
%%% End:
