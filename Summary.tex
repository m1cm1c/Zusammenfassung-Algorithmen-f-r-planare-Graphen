\documentclass[a4paper,11pt]{report}
\usepackage[T1]{fontenc}
\usepackage[utf8]{inputenc}
\usepackage[german]{babel}
\usepackage{lmodern}
\usepackage{mathtools}
\usepackage{graphicx}
\usepackage{grffile}
\usepackage{float}
\usepackage[a4paper, total={6in, 10in}]{geometry}
\usepackage[colorlinks]{hyperref}
\usepackage{csquotes}
\usepackage[acronym,toc]{glossaries}
\usepackage{enumitem}
\usepackage{listings}
\usepackage[section]{placeins}
\usepackage{amsmath}

\title{Zusammenfassung Algorithmen für planare Graphen}
\author{Christoph Michelbach}


\makeatletter
\hypersetup{
    colorlinks,
    citecolor=black,
    filecolor=black,
    linkcolor=blue,
    urlcolor=black,
    pdftitle={\@title},
    pdfauthor={\@author},
    bookmarks=true
}

\newcommand{\fig}[2]{
    \begin{figure}[h]
        \begin{center}
            \includegraphics[width=#2\textwidth]{Resources/Included Graphics/#1}
            \caption{#1}
            \label{fig:#1}
        \end{center}
    \end{figure}
}

\newcommand{\glsquotation}[1]{
    \paragraph{\gls{#1}:} \glsentrydesc{#1}.
}


\makeglossaries


\newglossaryentry{exampleEntry}
{
    name=example entry,
    plural=example entries,
    description={This is an example glossary entry},
}


\begin{document}
\maketitle
\tableofcontents

\chapter{Einführung}
$K_a$ ist ein vollstd. verbundener Graph aus $a$ Knoten.\\

$K_{a, b}$ ist ein Graph, dessen Knoten aus disjunkten Mengen $M_a$ und $M_b$ der Kardinalitäten $a$ bzw. $b$ bestehen, so dass alle Knoten aus $M_a$ mit allen Knoten aus $M_b$ verbunden sind, es aber keine keine weiteren Verbindungen gibt.\\

Einen Graphen auf Planarität zu prüfen, ist in $\mathcal{O}(n)$ möglich. Das liegt mit daran, dass die Anzahl der Kanten von planaren Graphen in $\mathcal{O}(n)$ liegt.\\

\paragraph{Satz von Koratowski} Ein Graph ist planar, gdw. er weder $K_5$ noch $K_{3, 3}$ als Teilgraphen enthält.\\

Die Einbettungen von Graphen sind verschieden, gdw. ihre Fasetten verschieden sind.
\fig{Bsp. fuer untersch. Fasetten: Die Fasette der oberen und der linken unteren Darst. sind gleich, die der rechten unteren aber nicht}{.7}



\glsaddallunused
\printglossaries

\end{document}


%%% Local Variables:
%%% mode: latex
%%% TeX-master: t
%%% End:
