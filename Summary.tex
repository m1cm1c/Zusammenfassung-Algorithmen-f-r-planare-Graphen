\documentclass[a4paper,11pt]{report}
\usepackage[T1]{fontenc}
\usepackage[utf8]{inputenc}
\usepackage[german]{babel}
\usepackage{lmodern}
\usepackage{mathtools}
\usepackage{graphicx}
\usepackage{grffile}
\usepackage{float}
\usepackage[a4paper, total={6in, 10in}]{geometry}
\usepackage[colorlinks]{hyperref}
\usepackage{csquotes}
\usepackage[acronym,toc]{glossaries}
\usepackage{enumitem}
\usepackage{listings}
\usepackage[section]{placeins}
\usepackage{amsmath}
\usepackage{amssymb}
\usepackage{parskip}

\title{Zusammenfassung Algorithmen für planare Graphen}
\author{Christoph Michelbach}


\makeatletter
\hypersetup{
    colorlinks,
    citecolor=black,
    filecolor=black,
    linkcolor=blue,
    urlcolor=black,
    pdftitle={\@title},
    pdfauthor={\@author},
    bookmarks=true
}

\newcommand{\fig}[2]{
    \begin{figure}[h]
        \begin{center}
            \includegraphics[width=#2\textwidth]{Resources/Included Graphics/#1}
            \caption{#1}
            \label{fig:#1}
        \end{center}
    \end{figure}
}

\newcommand{\glsquotation}[1]{
    \paragraph{\gls{#1}:} \glsentrydesc{#1}.
}


\makeglossaries


\newglossaryentry{geradeMenge}
{
    name=gerade Menge,
    description={Eine Knotenmenge, in der alle Knoten geraden Grad haben}
}

\newglossaryentry{perfectMatching}
{
    name=perfect Matching,
    description={Ein Matching, bei dem jeder Knoten des Graphen zu einer Kante des Matchings $M$ inzident ist. Das ist äquivalent zu $|M| = \frac{n}{2}$}
}

\newglossaryentry{nearPerfectMatching}
{
    name=near perfect Matching,
    description={Ein Matching, bei dem maximal 1 Knoten des Graphen nicht zu einer Kante des Matchings inzident ist}
}



\begin{document}
\maketitle
\tableofcontents

\chapter{Einführung}
Einen Graphen auf Planarität zu prüfen, ist in $\mathcal{O}(n)$ möglich. Das liegt mit daran, dass die Anzahl der Kanten von planaren Graphen in $\mathcal{O}(n)$ liegt.\\

\paragraph{Satz von Kuratowski} Ein Graph ist planar, gdw. er weder $K_5$ noch $K_{3, 3}$ als Teilgraphen enthält.\\

Die Einbettungen von Graphen sind verschieden, gdw. ihre Facetten verschieden sind.
\fig{Bsp. fuer untersch. Facetten: Die Facetten der oberen und der linken unteren Darst. sind gleich, die der rechten unteren aber nicht}{.7}

\paragraph{Jordan-Kurven} sind stetige, sich selbst nicht kreuzende Kurven.\\


\section{Dualgraph $G^*$}
Der Dualgraph zum pl. Graph $G = (V, E)$ mit pl. Einbettung $\mathcal{F}$ ist $G^* = (V^*, E^*)$, wobei für jede Facette $f \in \mathcal{F}$ ein Knoten $v_f \in V^*$ existiert und für jede Kante $e \in E$ eine Kante $e^* \in E^*$, die die beiden Knoten in $G^*$ verbindet, die zu den beiden Facetten in $\mathcal{F}$ gehören, an die $e$ angrenzt bzw. den Knoten mit sich selbst verbindet, der zu der Facette gehört, an die $e$ angrenzt.\\

\begin{itemize}
    \item $G^*$ ist auch planar.
    \item $G^*(G^*) = G$ (bzgl. kanonischer pl. Einbettung von $G$ bzw. $G^*$.
    \item $G^*$ kann Multigraph sein.
\end{itemize}

Für pl. Graphen $G$ mit bel. pl. Einbettung $\mathcal{F}$ gilt: Ein {\bf Schnitt} (als Kantenmenge) in $G$ entspricht einem {\bf Kreis} in $G^*$.


\section{Planare Einbettung}
Eine planare Einbettung zerlegt $\mathbb{R}^2$ in {\bf Facetten} (Gebiete, Flächen). Die Facettenmenge wird mit $\mathcal{F}$ bezeichnet. Die planare Einbettung ist bereits durch ihre Facettenmenge bzw. die Reihenfolge der Kurven um jede Fläche bzgl. Reihenfolge der Adjazenzlisten der Knoten festgelegt.


\section{Kombinatorische Einbettung}
Die kombinatorische Einbettung steht gegenüber der {\bf geometrischen Einbettung} (mit konkreten Koordinaten der Punkte). Die kombinatorische Einbettung wird durch die zyklische Reihenfolge der Kanten um die Knoten ausgezeichnet. Man sagt dann auch, dass 2 Einbettungen das selbe Rotationssystem haben.


\section{Satz von Euler (1750)}
$G = (V, E)$ ist ein pl. Graph mit bel. pl. Einbettung $\mathcal{F}$. $n \coloneqq |V|$, $m \coloneqq |E|$, $f \coloneqq |\mathcal{F}|$\\

In einem zusammenhängenden, planaren, einfachen (nicht Multigraph), nichtleeren Graphen gilt für jede planare Einbettung: $n - m + f = 2$

\section{Minoren}
Ein ungerichteter Graph $H$ ist ein {\bf Minor} von $G = (V, E)$, gdw. man $H$ aus $G$ erhalten kann, indem man ausschließlich diese Operationen anwendet:

\begin{itemize}
    \item Knoten löschen.
    \item Kanten löschen.
    \item Kanten kontrahieren. Dabei werden Knoten vereinigt, wobei der neu entstandene Knoten die Kanten der in die Vereinigung eingegangenen Knoten hat.
\end{itemize}

Pl. Graphen sind unter Minorenbildung abgeschlossen.


\chapter{Vorgabe-Graphen}
\section{$K$-Graphen}
\subsection{$K_a$}
$K_a$ ist ein vollstd. verbundener Graph aus $a$ Knoten. Beachte: ``vollständig verbundener Graph'' bedeutet, dass jeder Knoten mit {\bf jedem anderen} Knoten verbunden ist. Schlingen sind nicht enthalten.


\subsection{$K_{a, b}$}
$K_{a, b}$ ist ein Graph, dessen Knoten aus disjunkten Mengen $M_a$ und $M_b$ der Kardinalitäten $a$ bzw. $b$ bestehen, so dass alle Knoten aus $M_a$ mit allen Knoten aus $M_b$ verbunden sind, es aber keine keine weiteren Verbindungen gibt.

Speziell wird $K_{3, 2}$ auch ``$\Theta$-Graph'' genannt, da er gezeichnet einem $\Theta$ ähnelt.


\section{$T$-Graphen}
\subsection{$T_n$}
Die Knotenmenge von $T_n$ besteht genau aus den zweielementigen Teilmengen der Menge $\{1, \ldots, n\}$. Zwei Knoten sind genau dann durch eine Kante verbunden, wenn der Schnitt der zugehörigen Mengen nicht leer ist.


\section{$Q$-Graphen}
\subsection{$Q_n$}
$Q_n$ ist ein Würfel. Die Knoten sind Binärzahlen der Länge $n$. Eine Kante verläuft zwischen 2 Knoten gdw. diese sich in genau 1 bit unterscheiden.


\chapter{Graph-Operationen}
\paragraph{Hinweis:} Bezüglich weiteren Graph-Operationen, siehe auch Kapitel \ref{ch:DefVonSubgrUntertMinor}.


\section{Knotenteilung}
1 Knoten wird in 2 unterteilt. Die beiden entstandenen Knoten sind dann inzident zu genau den Knoten, zu denen der originale Knoten inzident war. Ob diese Knoten dann verbunden sind, hängt davon ab, was man damit bezweckt.


\chapter{Definition von ``Subgraph'', ``Unterteilung'', und ``Minor''}
\label{ch:DefVonSubgrUntertMinor}
\section{Subgraph}
$G' = (V', E')$ heißt Subgraph von $G = (V, E)$, gdw. $V' \subseteq V$ und $E' \subseteq E$.


\subsection{Knoteninduzierter Subgraph}
$G' = (V', E')$ heißt knoteninduzierter Subgraph von $G = (V, E)$, gdw. $V' \subseteq V$ und $E' = \{ \{u, v\} \in E : u, v \in V' \}$.


\section{Unterteilung}
$G' = (V', E')$ heißt Unterteilung von $G = (V, E)$, gdw. $G'$ aus $G$ entsteht, indem Kanten aus $G$ ($e \in E$) durch einfache Wege in $G'$ ersetzt werden. Dabei werden neue Knoten in den Graphen eingefügt. Diese neuen Knoten haben alle Grad 2.


\section{Topologischer Minor}
$H$ heißt (topologischer) Minor von $G$, gdw. $G$ eine Unterteilung von $H$ als Subgraph enthält.

In der Vorlesung werden topologische Minoren meist nur ``Minoren'' genannt.

Äquivalent dazu: $H$ ist (topologischer) Minor von $G$, gdw. $H$ aus $G$ durch Löschen von Knoten oder Kanten, oder Kontraktion von inzidenten Kanten entsteht, wobei bei der Kontraktion hier nur Knoten mit Grad 2 gelöscht werden dürfen (man kann aber vorher Kanten entfernen, um den Grad zu reduzieren).


\chapter{Arten von Graphen}
\section{Minor-minimale (mm.) nichtplanare (npl.) Graphen}
$G$ ist mm. npl., gdw. $G$ npl. ist, aber jeder echte Minor von $g$ pl. ist.

Offensichtlich sind $K_5$ und $K_{3, 3}$ mm. npl. Graphen.

Außerdem gilt:
\begin{itemize}
    \item Wenn $G$ npl., dann enthält $G$ mm. npl. Graphen als Minoren.
    \item Mm. npl. Graphen haben Mindestgrad 3.
\end{itemize}


\chapter{Separatoren in planaren Graphen}
Sei $G = (V, E)$ zusammenhängend. $S \subseteq V$ heißt Separator von $G$, gdw. der durch $V \setminus S$ induzierte Teilgraph von $G$ nicht zusammenhängend ist.

Knoten $u, v \in V$ werden durch $S$ getrennt, gdw. $u$ und $v$ in verschiedenen Zusammenhangskomponenten des durch $V \setminus S$ ind. Teilgraphen sind.

\paragraph{Interessant:} $|S|$ möglichst klein; und für $V_1, V_2 \subseteq V \setminus S$, $S$ trennt $V_1$ und $V_2$; und $V_1 \cup V_2 \cup S = V$; $V_1$ und $V_2$ möglichst ``balanciert''.


\section{Planar Separator Theorem}
Für $G = (V, E)$ pl. mit $n = |V| \geq 5$ ex. Partitionen $S, V_1, V_2$ von $V$, so dass:
\begin{itemize}
    \item $|V_1|, |V_2| \leq \frac{2}{3} \cdot n$
    \item $S$ trennt $V_1$ und $V_2$.
    \item $|S| \leq 4 \cdot \sqrt{n}$
\end{itemize}

S kann in Linearzeit ($\in \mathcal{O}(n)$) berechnet werden.


\chapter{Matchings}
Sei $G = (V, E)$. $M \subseteq E$ heißt {\bf Matching}, gdw. in M keine 2 Kanten existieren, die den selben Knoten haben.

Das Gewicht eines Matchings ist definiert als: $\text{w}(M) \coloneqq \sum_{e \in M} \text{w}(e)$

$P$ heißt bzgl. einem Matching $M$ \textbf{alternierender Weg}, gdw. $P$ ein einfacher Weg oder einfacher Kreis ist, und seine Kanten abwechseln in $M$ und $E \setminus M$ liegen.

Ein alternierender Weg $P$ heißt bezüglich $M$ \textbf{erhöhend}, gdw.
\[
    M' \coloneqq M \Delta P = (M \cup P) \setminus (M \cap P)
\]
ein Matching in $G$ mit $\text{w}(M') > \text{w}(M)$ ist.


\chapter{Maximale Flüsse}
$E(S, V \setminus S)$ ist ein $st$-Schnitt. Diese Notation bedeutet: Alle Kanten, die von $S$ nach $V \setminus S$ führen.


\section{Gewicht}
\[ w(\phi) \coloneqq \sum_{e \in E(S, V \setminus S)} \phi(e) - \sum_{e \in E(V \setminus S, S)} \phi(e) \]

Es gilt:

\[ w(\phi) \leq c(S, V \setminus S) \]

Ein $st$-Fluss $\phi$ mit $w(\phi) = c(S, V \setminus S)$ ist somit maximal.


\chapter{Menger-Problem}
\fig{Definition von Kappa beim Menger-Problem}{.7}

$\kappa_G(u, v)$ (Definition: Siehe Abbildung \ref{fig:Definition von Kappa beim Menger-Problem}.) heißt Knotenzug von $u$ und $v$.

\fig{Definition von Lambda beim Menger-Problem}{.6}


\section{Satz von Menger}
Für $G = (V, E)$, $s, t \in V$, $s \neq t$, ($\{s, t\} \not \in E$ im Fall i) gilt:

\begin{enumerate}
    \item $\kappa_G(s, t) \geq \kappa$, gdw. es mind. $\kappa$ intern knotendisjunkte $s$-$t$-Wege gibt.
    \item $\lambda_G(s, t) \geq \kappa$, gdw. es mind. $\kappa$ kantendisjunkte $s$-$t$-Wege gibt.
\end{enumerate}

\section{Menger-Problem}
Gegeben sei ein Graph $G = (V, E)$ und Knoten $s, t \in V$. Finde eine maximale Anzahl paarweise kanten- bzw. intern knotendisjunkter Wege, die $s$ und $t$ verbinden.


\section{Kantendisjunkter Menger-Algorithmus}
\begin{enumerate}
    \item Ersetze in Linearzeit $G = (V, E)$ durch den gerichteten Graphen $\vec G = (V, \vec E)$, der entsteht, indem jede Kante $\{u, v\} \in E$ ersetzt wird, durch die beiden gerichteten Kanten $(u, v), (v, u) \in \vec E$.
    \item Berechne in Linearzeit eine Menge geeigneter, einfacher, kantendisjunkter, gerichteter Kreise $\vec C_1, \ldots, \vec C_l$, und betrachte den Graphen $\vec G_C$, der aus $\vec G$ entsteht, indem alle Kanten, die auf einem $\vec C_i$ liegen, umgedreht werden.
    \item Berechne in Linearzeit eine maximale Anzahl von kantendisjunkten, gerichteten $s$-$t$-Wegen in $\vec G_C$.
    \item Berechne in Linearzeit aus der Menge der kantendisjunkten $s$-$t$-Wege in $\vec G_C$ eine Menge kantendisjunkter $s$-$t$-Wege in $G$ gleicher Kardinalität.
\end{enumerate}


\chapter{Graph-Probleme}
\section{Listenfärbungsproblem}
\paragraph{Gegeben:} $G = (V, E)$, $k \in \mathbb{N}$

\paragraph{Problem:} Gibt es für jede Zuordnung $S_v, v \in V$ mit $|S_v| = k$ ($S_v$ ist eine Liste von Farben für $v$.) eine korrekte Knotenfärbung, bei der die Farbe eines jeden Knoten $v \in V$ aus seiner Liste $S_v$ stammt?

\paragraph{Motivation:} Frequenzzuweisung in Kommunikationsnetzen.


\section{Minimum balanced Separator Problem}
Gegeben $G = (V, E)$, finde Partitionen $V_1, V_2, S$ von $V$, so dass $S$ Separator, der $V_1$ und $V_2$ trennt, $|S|$ minimal, und $V_1, V_2 \leq \alpha \cdot |V|$ für $\frac{1}{2} \leq \alpha < 1$ konstant.

Dieses Problem ist im Allgmeinen NP-schwer. Ob es bei pl. Gr. für $\alpha = \frac{1}{2}$ NP-schwer ist, ist offen.


\section{Mixed-Max Cut Problem}
Gegeben $G = (V, E)$, $w : E \rightarrow \mathbb{R}$. Finde nichttrivialen Schnitt $S \subset E$ maximalen Gewichts.

Dieses Problem ist i.A. {\bf NP-schwer}.

Der Spezialfall, den wir betrachten, ist, dass $w : E \rightarrow \mathbb{R}^+$. In allgemeinen Graphen funktioniert das in $\mathcal{O}(n \cdot m + n^2 \cdot \log(n))$.


\subsection{Algorithmus für Mixed-Max Cut in pl. Gr.}
\paragraph{Laufzeit:} $\mathcal{O}(n^{\frac{3}{2}} \cdot \log(n))$

\begin{enumerate}
    \item Trianguliere $G$.
    \item Berechne Dualgraph $G^* = (V^*, E^*)$ und induzierte Gewichtsfunktion $w^* : E^* \rightarrow \mathbb{R}$.
    \item Konstruiere zu $G^*$ einen Graphen $G' = (V', E')$, so dass ein \glslink{perfectMatching}{perfektes Matching} minimalen Gewichts $M$ in $G'$ eine \gls{geradeMenge} maximalen Gewichts $M^*$ in $G^*$ induziert. Berechne $M$ und entspr. Menge $M^*$. Falls $M^* \neq \emptyset$, gib duale Menge zu $M^*$ aus.
    \item Sonst (d.h. falls $M^* = \emptyset$: Sonderfall. (Siehe Script.)
\end{enumerate}


\chapter{Problem von Okamura und Seymour}
\paragraph{Geradheitsbedingung}
Für jeden Knoten gilt: Der Grad des Knotens minus wie oft dieser Knoten als Terminal vorkommt, ist gerade.

\paragraph{Kapazitätsbedingung} $\text{fcap}(X) = \text{cap}(X) - \text{dens}(X) \geq 0$\\

Ist die Geradheitsbedingung erfüllt, so ist die Kapazitätsbedingung notwendig und hinnreichend für die Existenz einer Lösung. Das ist der \textbf{Satz von Okamura und Seymour}.

Ist die Geradheitsbedingung nicht erfüllt, so ist die Kapizitätsbedingung notwendige Bedingung für die Existenz einer Lösung, aber nicht hinreichende.



\chapter{Lemmas}
\begin{itemize}
    \item Ein pl. Graph mit mind. 3 Knoten hat höchstens $3n-6$ Kanten.
    \item Ein maximaler pl. Graph (ein Graph, zu dem man keine Kante hinzufügen kann, ohne Planarität zu verletzen) mit $n$ Knoten, hat die max. Anzahl an Knoten unter allen pl. Graphen mit $n$ Knoten.
    \item Sei $G$ pl. mit mind. 3 Knoten. $d_{max}(G)$ bezeichne den Maximalgrad in $G$ und $n_j$ bezeichne die Anzahl Knoten in $G$ mit Grad $i$. Dann gilt: \[6 n_0 + 5 n_1 + 4 n_2 + 3 n_3 + 2 n_4 + n_5 \geq n_7 + 2 n_8 + 3 n_9 + ... + (d_{max}(G) - 6) \cdot n_{d_{max}(G)}\]
    \item Wenn $G$ pl. Gr. und $c$ einfacher Kreis in $G$ ist und keine 2 Knoten auf $c$ in $G - E(c)$ verbunden sind (Weg zwischen beiden existiert), dann besitzt G eine planare Einbettung, bei der $c$ eine Facette begrenzt.
    \item Jeder pl. Gr. ist 4-färbbar, aber nicht jeder pl. Gr. ist 3-färbbar.
    \item Jeder pl. Gr. ist 5-listenfärbbar, aber nicht jeder pl. Gr. ist 4-listenfärbbar.
    \item Sei $G = (V, E)$ pl. zusammenhängend; $|V| = n \geq 4$; und $T = (V(T), E(T))$ aufgespannter Baum in $G$ der Höhe $h$. Dann ex. Partitionen $S$, $V_1$, $V_2$ von $V$, mit:
    \begin{itemize}
        \item $|V_1|, |V_2| \leq \frac{2}{3} \cdot n$
        \item $S$ Separator, der $V_1$ und $V_2$ trennt
        \item $|S| \leq 2 \cdot h + 1$
    \end{itemize}
    Die Partitionen können in $\mathcal{O}(n)$ berechnet werden.

    \item Das {\bf Knotendefizit} ist definiert als $\text{def}(v) \coloneqq 6 - \text{deg}(v)$. F.a. max. pl. Gr. gilt: $$\sum_{v \in V} \text{def}(v) = 12$$
\end{itemize}


\glsaddallunused
\printglossaries

\end{document}


%%% Local Variables:
%%% mode: latex
%%% TeX-master: t
%%% End:
